\documentclass[10pt,twocolumn,letterpaper]{article}
\DeclareSymbolFont{AMSb}{U}{msb}{m}{n}
\DeclareMathAlphabet{\mathbbm}{U}{bbm}{m}{n}
\title{Category Theory Cheat Sheet}
%\author{Nathaniel Wesley Filardo}

\usepackage{amsmath,amssymb,amsthm,latexsym}
\usepackage{fancyhdr}
\usepackage[tiny,center,compact,sc]{titlesec}
\usepackage[cm]{fullpage}
\usepackage{pstricks}
\usepackage{graphicx}
\usepackage{verbatim}
\usepackage{bm}
\usepackage{ifthen}
\usepackage{epsfig}
\usepackage[all]{xypic}
\usepackage{textcomp}
\usepackage{url}
\usepackage{multirow}
\usepackage{hyperref}
\usepackage{breakurl}
\usepackage{enumitem}
\setlist{nolistsep}

\renewcommand{\baselinestretch}{0.9}

%\newtheorem{thm}{Thm}[section]
%\newtheorem{dfn}{Def}[section]

\setlength{\parindent}{0pt}
\setlength{\parskip}{2pt}

% http://www.latex-community.org/forum/viewtopic.php?f=46&t=3837&start=0#p15112
\makeatletter
\g@addto@macro\normalsize{%
\setlength\abovedisplayskip{0pt}%
\setlength\abovedisplayshortskip{0pt}%
\setlength\belowdisplayskip{0pt}%
\setlength\belowdisplayshortskip{0pt}%
}
\makeatother

% http://comments.gmane.org/gmane.comp.tex.xy-pic/223
% fixes {>->} having the tail overlap the item.
\newdir{ >}{{}*!/-2.6667\jot/\dir{>}}


%Scalable bracket-like
\newcommand{\paren}[1]{\left({#1}\right)}
\newcommand{\brak}[1]{\left[{#1}\right]}
\newcommand{\abs}[1]{\left\lvert{#1}\right\rvert}
\newcommand{\ang}[1]{\left\langle{#1}\right\rangle}
\newcommand{\set}[1]{\left\{{#1}\right\}}

%Mathematics
\newcommand{\condexp}[1]{\ifthenelse{\equal{#1}{false}}{}{^{#1}}}
\newcommand{\dd}[3][false]{\frac{d\condexp{#1}{#2}}{d{#3}\condexp{#1}}}
\newcommand{\pd}[3][false]{\frac{\partial\condexp{#1}{#2}}{\partial{#3}\condexp{#1}}}

\newcommand{\ifrac}[2]{{#1}/{#2}}

%Quantum Mechanics
\newcommand{\ket}[1]{\left\lvert{#1}\right\rangle}
\newcommand{\bra}[1]{\left\langle{#1}\right\rvert}
\newcommand{\braket}[2]{\left\langle{#1}\middle\vert{#2}\right\rangle}
\newcommand{\Braket}[3]{\left\langle{#1}\middle\vert{#2}\middle\vert{#3}\right\rangle}
\newcommand{\dyad}[2]{\left\lvert{#1}\middle\rangle\middle\langle{#2}\right\rvert}

\DeclareMathOperator{\mm}{\mid\mid}
\newcommand{\natto}{\overset{\cdot}{\to}}

\newcommand{\defn}[1]{{\bf #1}}

\begin{document}

References are to Ji\v{r}\'i Ad\'amek, Horst Herrlich, George
E. Strecker's {\em Abstract and Concrete Categories: The Joy of Cats}.
Notation follows theirs with some contamination from Awodey and Pierce's
texts.

Entries within each section are roughly sorted by definition, alphabetically.

\section{Basics}

  A \defn{category} $\mathbf{C}$ (\S3.1) is a quadruple
  $(\mathcal{O},\mbox{hom},id,\circ)$ with
  \begin{itemize}
    \item A collection of objects $\mathcal{O}$
  \item For each pair of objects $A,B$, a (disjoint) collection of arrows
    from \defn{domain} $A$ to \defn{codomain} $B$,
    $\mbox{hom}(A,B)$ (also written $\mathbf{C}(A,B)$).
    \item An associative arrow composition operator $\circ$.
    \item Identity arrows ($id_A$) on each object $A$, unit of $\circ$
  \end{itemize}

  The \defn{dual} (\S3.5) category $\mathbf{A}^\text{op}$ which
    exchanges domains and codomains of arrows in $\mathbf{A}$.

  A predicate $P$ is \defn{essentially unique} (\S7.3) if it is unique up to
  isomorphism:
  \begin{itemize}
    \item If both $PA$ and $PB$, then $A \simeq B$
    \item If $PA$ and $A \simeq B$, then $PB$.
  \end{itemize}


  A category is$\dots$
  \begin{itemize}
    \item \defn{balanced} if all bi are iso (\S7.49.2)
    \item \defn{discrete} if all morphisms are identities. (\S3.26.1)
    \item \defn{thin} if $\forall_{A,B} \mbox{hom}(A,B) \simeq \set{*}$. (\S3.26.2)
  \end{itemize}

\section{Kinds of Objects}

  $C$ is a \defn{coseparator} if $\forall_{f,g : B \to A} . f \ne g
  \Rightarrow \exists_{h : A \to C} . h \circ f \ne h \circ g$. (\S7.17)
  (Contrast monomorphism.)

  An object $0$ is \defn{initial} if $\forall_B . \exists! f_B : 0 \to B$.
  Initial objects are essentially unique. (\S7.1)

  $S$ is a \defn{separator} if $\forall_{f,g : A \to B} . f \ne g
  \Rightarrow \exists_{h : S \to A} . f \circ h \ne g \circ h$. (\S7.10)
  (Contrast epimorphism.)

  $S$ is a separator iff $\mbox{hom}(S,-)$ is faithful. (\S7.12)

  A set of objects $\mathcal{T}$ is a \defn{separating set} if
  $\forall_{f,g : A \to B} . f \ne g \Rightarrow \exists{S \in \mathcal{T},
  h : S \to A} . f \circ h \ne g \circ h$. (\S7.14)

  An object $1$ is \defn{terminal} if $\forall_A . \exists! f_A : A \to 1$.
  Terminal objects are essentially unique. (\S7.4)

  An object that is both initial and terminal is called a \defn{zero}.
  (\S7.7)

\section{Kinds of Arrows}

  $e$ is an \defn{epimorphism} (\S7.39) (the dual of a monomorphism) if
  $ie = je \Rightarrow i = j$ in
    \[\xymatrix{A \ar@{->>}[r]^e & B \ar@<1ex>[r]^{i} \ar@<-1ex>[r]_j & C} \]
  If $f$ and $g$ are epis, then so is $g \circ f$; if $g \circ f$ is epi,
  then so is $g$. (\S7.41)  Epis generalize \defn{surjection} 

  $(E,e)$ is an \defn{equalizer} (\S7.51) of $f,g$ iff $fe = ge$ and
     \[\forall_{Z,z . zf = zg} \exists!_u eu = z \quad
     \xymatrix{
     Z \ar@{..>}[r]^u \ar@/_1pc/[rr]^{z} & E \ar[r]^e & A \ar@<1ex>[r]^f \ar@<-1ex>[r]_g & B \\
     }\]
  Equalizers are essentially unique. (\S7.53)

  A mono $m$ is a \defn{extremal} (\S7.61) if $e$ epic and
  $m = f \circ e$ implies that $e$ iso.

  Let $G: \mathbf{A} \to \mathbf{B}$ and $B \in \mathbf{B}$.  A
  \defn{$G$-structured arrow with domain $B$} is a pair $(f : B \to GA, A)$.
  (\S8.30)  It is
  \begin{itemize}
    \item \defn{generating} if $\forall_{r,s : A \to A'} . Gr \circ f = Gs
      \circ f \implies r = s$
    \item \defn{extremally generating} if it is generating and $\forall_{m :
      A' \to A, m ~\text{mono}, (g,A')} . f = Gm \circ g \implies m ~\text{iso}$.
    \item \defn{$G$-universal for $B$} if $\forall_{(f', A')} .
    \exists!_{\check f} . f' = G{\check f} \circ f$.  That is,
    \[\xymatrix{
        B \ar[r]^f \ar@/_1.25pc/[rr]^{f'}
        & GA \ar@{.>}[r]^{G{\check f}}
        & GA'
        & A \ar@{.>}[r]^{\check f}
        & A'
    }\]
  \end{itemize}

  $f : A \to B$ is an \defn{isomorphism} if $\exists!_g . f \circ g = id_B
    ~\wedge~ g \circ f = id_A$. (\S3.8; ! in \S3.11)

  $f$ is a \defn{monomorphism} (\S7.32) if $mi = mj \Rightarrow i = j$ in
    \[\xymatrix{C \ar@<1ex>[r]^{i} \ar@<-1ex>[r]_j & A \ar@{{ >}->}[r]^m & B} \]
  If $f$ and $g$ are monos, then so is $g \circ f$; if $g \circ f$ is mono,
  then so is $f$. (\S7.34)  Monos generalize \defn{injection} in
  $\mathbf{Set}$.

  $f$ is a \defn{regular monomorphism} (\S7.56) if it is an equalizer of
  some pair of morphisms.

  $f : A \to B$ is a \defn{retraction} if $\exists_g . f \circ g = 1_B$.
  If $f$ and $g$ are retractions, then so is $g \circ f$; if $g \circ f$
  is a retraction, then so is $g$. (\S7.27)

  $f : A \to B$ is a \defn{section} if $\exists_g . g \circ f = 1_A$.
  (\S7.19)
  If $f$ and $g$ are sections, then so is $g \circ f$;
  if $g \circ f$ is a section, then so is $f$. (\S7.21)

  Several morphism properties combine in useful ways:
  \begin{itemize}
    \item mono, epi $\Rightarrow$ \defn{bimorphism} (\S7.49)
    \item section $\Rightarrow$ regular mono (\S7.35, \S7.59.1)
    \item regular mono $\Rightarrow$ extremal mono (\S7.59.2, \S7.63)
    \item retraction $\Rightarrow$ epi (\S7.42)
    \item mono, retraction $\Leftrightarrow$ isomorphism (\S7.36)
    \item section, epi $\Leftrightarrow$ isomorphism (\S7.43)
  \end{itemize}
  %(XXX stopped around \S7.60; there's more to be said)

\section{Functors}

  By default, functors $F,G : \mathbf{A} \to \mathbf{B}$.

  A \defn{(covariant) functor} $F$ (\S3.17) assigns to each
  $\mathbf{A}$-object a $\mathbf{B}$-object and to each
  $\mathbf{A}$-morphism a $\mathbf{B}$-morphism s.t. composition and
  identites are {\em preserved}.

  A \defn{contravariant functor} $F$ (\S3.20.5) is a (covariant) functor
  $\mathbf{A}^\text{op} \to \mathbf{B}$.

  A \defn{endofunctor} has $\mathbf{A} = \mathbf{B}$.  $F \circ F$ may be
  denoted $F^2$, etc. (\S3.23; ftn 15)

  Functors compose. (\S3.23)

  A functor $F$ is (\S3.27, \S3.33)
  \begin{itemize}
    \item \defn{amnestic} if $f$ is an identity iff $Ff$ is an identity.
    \item an \defn{equivalence} if it is full, faithful, and
      isomorphism-dense.
    \item an \defn{embedding} if it is injective on morphisms.
    \item \defn{faithful} if $\forall_{A,A'} . F\vert_{\mathbf{A}(A,A')}
      \subseteq \mathbf{B}(FA, FA')$ is injective.
    \item \defn{full} if $\forall_{A,A'} . F\vert_{\mathbf{A}(A,A')}$ surjective.
    \item \defn{isomorphism-dense} if $\forall_B . \exists_A . F(A) \simeq B$.
  \end{itemize}

  A \defn{natural transformation} $\tau : F \natto G$ assigns each
  $A \in \mathbf{A}$ to $\tau_A : FA \to GA$ s.t.
  $\forall_{f : A \to A'} . G f \circ \tau_A = \tau_{A'} \circ F f$. (\S6.1)

  There is special notation for functors applied to natural transformations
  and vice-versa (\S6.2): $(F\tau)_A = F(\tau_A)$ and $(\tau F)_A =
  \tau_{FA}$.

  All functors \defn{preserve} (in $\mathbf{A}$ implies in $\mathbf{B}$)$\dots$
    \begin{itemize}
      \item isomorphisms. (\S3.21)
      \item sections (\S7.22)
      \item retractions (\S7.28)
    \end{itemize}

  Representable functors preserve monos. (\S7.37.1)

  Some functors \defn{reflect} (in $\mathbf{B}$ implies in $\mathbf{A}$) useful properties:
  \begin{itemize}
      \item Full, faithful functors reflect sections (\S7.23) and retractions (\S7.29).
      \item Faithful functors reflect monos (\S7.37.2) and epis (7.44).
  \end{itemize}

\section{Sources and Sinks}

  A \defn{source} in category $\mathbf{A}$ indexed by $I$ is a pair $(A,
  \set{f_i : A \to A_i}_{i \in I})$.  This source has domain $A$ and
  codomain $\set{A_i}_{i\in I}$. (\S10.1)

  Given $(A,\set{f_i}_{i \in I})$ and
  $\{(A_i,\set{g_{ij}}_{j \in J_i})\}_{i \in I}$
  all sources, their \defn{composite} is $(A, \set{g_{ij} \circ f_i}_{i\in I,
  j\in J_i})$. (\S10.3)

  A \defn{mono-source} (\S10.5) is $(A,\set{f_i})$ s.t. \\ $\forall r,s: B \to A.
  \brak{\forall_{i\in I} . f_i \circ r = f_i \circ s} \Rightarrow r = s$.

\section{Concrete Categories}

  For this section, $\mathbf{A}$ is a \defn{concrete category} over
  $\mathbf{X}$ with \defn{forgetful} functor $U : \mathbf{A} \to \mathbf{X}$
  faithful, denoted $(\mathbf{A}, U)$.  (\S5.1.1)

  When $\mathbf{A} = \mathbf{X}$, $\mathbf{Alg}(U)$ has as objects
  $U$-\defn{algebra}s $(X \in \mathbf{X}, h : UX \to X)$ and morphisms
  $f : (X,h) \to (X',h')$ s.t. $f \circ h = h' \circ T(f)$. (\S5.37)

  If $\mathbf{X}$ is $\mathbf{Set}$, $\mathbf{A}$ is a \defn{construct}.
  (\S5.1.2)

  $(UA \overset{f}{\to} UB) \in \mathbf{X}$ \defn{is an $\mathbf{A}$-morphism}
  if $f$ has a unique $U$-preimage in $\mathbf{A}$. (\S5.3, \S6.22)

  %An object $A\in\mathbf{A}$ is
  %\dots\! if $\forall_{B \in \mathbf{A}}$, \dots is an $\mathbf{A}$ arrow.
  %\begin{itemize}
  %  \item \defn{discrete}, $(UA \to UB)$ (\S8.1)
  %  \item \defn{indiscrete}, $(UB \to UA)$ (\S8.3)
  %\end{itemize}

  A \defn{free object} $A \in \mathbf{A}$ is one with a ($U$-structured)
  universal arrow $(u,UA)$ in $B$. (\S8.22+\S8.30)

  %$f \in \mathbf{A}$ is \defn{initial} if $\forall_{C \in \mathbf{A}}$ $UC
  %\overset{f \circ g}{\to} UB$ is an $\mathbf{A}$-morphism implies that $UC
  %\overset{g}{\to} UA$ is an $\mathbf{A}$-morphism.

\section{Adjoints and Adjoint Situations}

  A functor $G : \mathbf{A} \to \mathbf{B}$ is \defn{adjoint} if
  $\forall_{B \in \mathbf{B}}$ there exists a $G$-structured universal
  arrow with domain $B$.  (\S18.1)

  Adjoints compose (\S8.5), preserve mono sources (\S8.6), and preserve
  limits (\S8.9)

  Given adjoint $G$ with $\eta_B : B \to G(A_B)$ the $G$-structured
  universal arrow with domain $B$, $\exists!_F$ such that $FB = A_B$ and
  $\eta : id_B \natto G \circ F$ is natural; further, there is a unique,
  natural $\epsilon : F \circ G \natto id_A$ with $G\epsilon \circ \eta G =
  id_G$ and $\epsilon F \circ F \eta = id_F$.  (\S19.1)

  $(\eta,\epsilon) : F \dashv G : \mathbf{A} \to \mathbf{B}$ is a
  \defn{adjoint situation} if the above relationships hold. (\S19.7)

  A \defn{monad} (\S20.1) on $\mathbf{X}$ is $(T : \mathbf{X} \to \mathbf{X},
  \eta : id_{\mathbf{X}} \natto T, \mu : T^2 \natto T)$ s.t.
  \[\forall_X \quad
  \xymatrix@R=10pt{
    T^3X \ar[r]^{T(\mu_X)} \ar[d]^{\mu_{TX}} & T^2X \ar[d]^{\mu_{X}} \\
    T^2X \ar[r]^{\mu_X}                 & TX
  } \quad \xymatrix@R=10pt{
    TX \ar[r]^{T(\eta_X)} \ar[dr]_{id_{TX}} & T^2X \ar[d]^{\mu_X} & TX \ar[l]_{\eta_{TX}} \ar[dl]^{id_{TX}} \\
        & TX & 
  }\]


\pagebreak\appendix\section{Examples To Jog Your Memory}

\subsection{$\mathbf{Mon}$}

  $(\set{*},\cdot,id_*)$ is a zero (both initial and terminal).

  Mono-epics are not isos ($(\mathbf{N},+,0) \to (\mathbf{Z},+,0)$).
  (Pierce:\S1.6.3)

\end{document}
