\documentclass[10pt,twocolumn,letterpaper]{article}
\DeclareSymbolFont{AMSb}{U}{msb}{m}{n}
\DeclareMathAlphabet{\mathbbm}{U}{bbm}{m}{n}
\title{Category Theory Cheat Sheet}
%\author{Nathaniel Wesley Filardo}

\usepackage{amsmath,amssymb,amsthm,latexsym}
\usepackage{fancyhdr}
\usepackage[tiny,center,compact,sc]{titlesec}
\usepackage[cm]{fullpage}
\usepackage{pstricks}
\usepackage{graphicx}
\usepackage{verbatim}
\usepackage{bm}
\usepackage{ifthen}
\usepackage{epsfig}
\usepackage[all]{xypic}
\usepackage{textcomp}
\usepackage{url}
\usepackage{multirow}
\usepackage{hyperref}
\usepackage{breakurl}

\renewcommand{\baselinestretch}{0.9}

%\newtheorem{thm}{Thm}[section]
%\newtheorem{dfn}{Def}[section]

\setlength{\parindent}{0pt}
\setlength{\parskip}{3pt}

%Scalable bracket-like
\newcommand{\paren}[1]{\left({#1}\right)}
\newcommand{\brak}[1]{\left[{#1}\right]}
\newcommand{\abs}[1]{\left\lvert{#1}\right\rvert}
\newcommand{\ang}[1]{\left\langle{#1}\right\rangle}
\newcommand{\set}[1]{\left\{{#1}\right\}}

%Mathematics
\newcommand{\condexp}[1]{\ifthenelse{\equal{#1}{false}}{}{^{#1}}}
\newcommand{\dd}[3][false]{\frac{d\condexp{#1}{#2}}{d{#3}\condexp{#1}}}
\newcommand{\pd}[3][false]{\frac{\partial\condexp{#1}{#2}}{\partial{#3}\condexp{#1}}}

\newcommand{\ifrac}[2]{{#1}/{#2}}

%Quantum Mechanics
\newcommand{\ket}[1]{\left\lvert{#1}\right\rangle}
\newcommand{\bra}[1]{\left\langle{#1}\right\rvert}
\newcommand{\braket}[2]{\left\langle{#1}\middle\vert{#2}\right\rangle}
\newcommand{\Braket}[3]{\left\langle{#1}\middle\vert{#2}\middle\vert{#3}\right\rangle}
\newcommand{\dyad}[2]{\left\lvert{#1}\middle\rangle\middle\langle{#2}\right\rvert}

\DeclareMathOperator{\mm}{\mid\mid}

\newcommand{\defn}[1]{{\bf #1}}

\begin{document}

Notation and references are to Ji\v{r}\'i Ad\'amek, Horst Herrlich, George
E. Strecker's {\em Abstract and Concrete Categories: The Joy of Cats}.

Entries within each section are roughly sorted by definition, alphabetically.

\section{Basics}

  A \defn{category} (\S3.1) is a quadruple
  $(\mathcal{O},\mbox{hom},id,\circ)$ with
  \begin{itemize}
    \item A collection of objects $\mathcal{O}$
  \item For each object $A,B$, a (disjoint) collection of arrows
    $\mbox{hom}(A,B)$ (from \defn{domain} to \defn{codomain}).
    \item An associative arrow composition operator $\circ$.
    \item Identity arrows ($id_A$) on each object $A$, unit of $\circ$
  \end{itemize}

  The \defn{dual} (\S3.5) category $\mathbf{A}^\text{op}$ which
    exchanges domains and codomains of arrows in $\mathbf{A}$.

  A predicate $P$ is \defn{essentially unique} (\S7.3) if it is unique up to
  isomorphism:
  \begin{itemize}
    \item If both $PA$ and $PB$, then $A \simeq B$
    \item If $PA$ and $A \simeq B$, then $PB$.
  \end{itemize}

  In a concrete category $(\mathbf{A}, U)$ over $\mathbf{X}$, $(UA
  \overset{f}{\to} UB) \in \mathbf{X}$ \defn{is an $\mathbf{A}$-morphism} if
  $f$ has a unique $U$-preimage in $\mathbf{A}$. (\S5.3, \S6.22)

\section{Predicates on Categories}

  A category is$\dots$
  \begin{itemize}
    \item \defn{balanced} if all bimorphisms are isomorphisms (\S7.49.2)
    \item \defn{discrete} if all morphisms are identities. (\S3.26.1)
    \item \defn{thin} if $\forall_{A,B} \mbox{hom}(A,B) \simeq \set{*}$. (\S3.26.2)
  \end{itemize}

  A \defn{concrete category} over $\mathbf{X}$ is a pair $(\mathbf{A},U :
  \mathbf{A} \to \mathbf{X}$) with $U$ faithful. (\S5.1.1)

  A \defn{construct} is a concrete category over $\mathbf{Set}$. (\S5.1.2)

\section{Kinds of Objects}

  $C$ is a \defn{coseparator} if $\forall_{f,g : B \to A} . f \ne g
  \Rightarrow \exists_{h : A \to C} . h \circ f \ne h \circ g$. (\S7.17)

  An object $A$ in a concrete category $\mathbf{A}$ over $\mathbf{X}$ is
  \dots\! if $\forall_{B \in \mathbf{A}}$, \dots is an $\mathbf{A}$ arrow.
  \begin{itemize}
    \item \defn{discrete}, $(UA \to UB)$ (\S8.1)
    \item \defn{indiscrete}, $(UB \to UA)$ (\S8.3)
  \end{itemize}

  An object $0$ is \defn{initial} if $\forall_B . \exists! f_B : 0 \to B$.
  Initial objects are essentially unique. (\S7.1)

  $S$ is a \defn{separator} if $\forall_{f,g : A \to B} . f \ne g
  \Rightarrow \exists_{h : S \to A} . f \circ h \ne g \circ h$. (\S7.10)

  $S$ is a separator iff $\mbox{hom}(S,-)$ is faithful. (\S7.12)

  A set of objects $\mathcal{T}$ is a \defn{separating set} if
  $\forall_{f,g : A \to B} . f \ne g \Rightarrow \exists{S \in \mathcal{T},
  h : S \to A} . f \circ h \ne g \circ h$. (\S7.14)

  An object $1$ is \defn{terminal} if $\forall_A . \exists! f_A : A \to 1$.
  Terminal objects are essentially unique. (\S7.4)

  An object that is both initial and terminal is called a \defn{zero}.
  (\S7.7)

\section{Kinds of Arrows}

  $e$ is an \defn{epimorphism} (\S7.39) if it is monic in $\mathbf{C}^{op}$,
          {\it i.e.,} if $ie = je \Rightarrow i = j$ in
    \[\xymatrix{A \ar@{->>}[r]^e & B \ar@<1ex>[r]^{i} \ar@<-1ex>[r]_j & C} \]
  If $f$ and $g$ are epis, then so is $g \circ f$; if $g \circ f$ is epi,
  then so is $g$. (\S7.41)

  $(E,e)$ is an \defn{equalizer} (\S7.51) of $f,g$ iff $fe = ge$ and
     \[\forall_{Z,z . zf = zg} \exists!_u eu = z \quad
     \xymatrix{
     Z \ar@{..>}[r]^u \ar@/_1pc/[rr]^{z} & E \ar[r]^e & A \ar@<1ex>[r]^f \ar@<-1ex>[r]_g & B \\
     }\]
  Equalizers are essentially unique. (\S7.53)

  A mono $m$ is a \defn{extremal} (\S7.61) if $e$ epic and
  $m = f \circ e$ implies that $e$ iso.

  Let $G$ be a functor $\mathbf{A} \to \mathbf{B}$ and $B$ a
  $\mathbf{B}$-object. A \defn{$G$-structured arrow with domain $B$}
  is a pair $(f : B \to GA, A)$. (\S8.30)  It is
  \begin{itemize}
    \item \defn{generating} if $\forall_{r,s : A \to A'} . Gr \circ f = Gs
      \circ f \implies r = s$
    \item \defn{extremally generating} if it is generating and $\forall_{m :
      A' \to A, m ~\text{mono}, (g,A')} . f = Gm \circ g \implies m ~\text{iso}$.
    \item \defn{$G$-universal for $B$} if $\forall_{(f', A')} .
    \exists!_{\check f} . f' = G{\check f} \circ f$.
  \end{itemize}

  $f : A \to B$ is an \defn{isomorphism} if $\exists!_g . f \circ g = id_B
    ~\wedge~ g \circ f = id_A$. (\S3.8; ! in \S3.11)

  $f$ is a \defn{monomorphism} (\S7.32) if $mi = mj \Rightarrow i = j$ in
    \[\xymatrix{C \ar@<1ex>[r]^{i} \ar@<-1ex>[r]_j & A \ar@{>->}[r]^m & B} \]
  If $f$ and $g$ are monos, then so is $g \circ f$; if $g \circ f$ is mono,
  then so is $f$. (\S7.34)

  $f$ is a \defn{regular monomorphism} (\S7.56) if it is an equalizer of
  some pair of morphisms.

  $f : A \to B$ is a \defn{retraction} if $\exists_g . f \circ g = 1_B$.
  If $f$ and $g$ are retractions, then so is $g \circ f$; if $g \circ f$
  is a retraction, then so is $g$. (\S7.27)

  $f : A \to B$ is a \defn{section} if $\exists_g . g \circ f = 1_A$.
  (\S7.19)
  If $f$ and $g$ are sections, then so is $g \circ f$;
  if $g \circ f$ is a section, then so is $f$. (\S7.21)


  Several morphism properties combine in useful ways:
  \begin{itemize}
    \item mono, epi $\Rightarrow$ \defn{bimorphism} (\S7.49)
    \item retraction $\Rightarrow$ epi (\S7.42)
    \item section, retraction $\Leftrightarrow$ isomorphism (\S7.26)
    \item mono, retraction $\Leftrightarrow$ isomorphism (\S7.36)
    \item section, epi $\Leftrightarrow$ isomorphism (\S7.43)
    \item section $\Rightarrow$ regular mono (\S7.35; regular \S7.59.1)
    \item regular mono $\Rightarrow$ extremal mono (\S7.59.2; extremal \S7.63)
  \end{itemize}
  (XXX stopped around 7.60; there's more to be said)

\subsection{Arrows in Concrete Categoies}

  For this section, $\mathbf{A}$ is a concrete category over $\mathbf{X}$.

  $f \in \mathbf{A}$ is \defn{initial} if $\forall_{C \in \mathbf{A}}$ $UC
  \overset{f \circ g}{\to} UB$ is an $\mathbf{A}$-morphism implies that $UC
  \overset{g}{\to} UA$ is an $\mathbf{A}$-morphism.

\section{Functors}

  A \defn{(covariant) functor} $F : \mathbf{A} \to \mathbf{B}$ (\S3.17)
  assigns to each $\mathbf{A}$-object a $\mathbf{B}$-object and to each
  $\mathbf{A}$-morphism a $\mathbf{B}$-morphism s.t. composition and
  identites are {\em preserved}.

  A \defn{contravariant functor} $F : \mathbf{A} \to \mathbf{B}$ (\S3.20.5)
  is a functor from $\mathbf{A}^\text{op} \to \mathbf{B}$.

  A functor is (\S3.27)
  \begin{itemize}
    \item \defn{amnestic} if $f$ is an identity iff $Ff$ is an identity.
    \item an \defn{embedding} if it is injective on morphisms.
    \item \defn{faithful} if $\forall_{A,A'}$ the restriction $F\vert_{\mbox{hom}_A(A,A')}
      \subseteq \mbox{hom}(FA, FA')$ is injective.
    \item \defn{full} if said restrictions are surjective.
  \end{itemize}

  A functor $F : \mathbf{A} \to \mathbf{B}$ is (\S3.33)
  \begin{itemize}  
    \item an \defn{equivalence} if it is full, faithful, and
      isomorphism-dense.
    \item \defn{isomorphism-dense} if $\forall_B . \exists_A . F(A) \simeq B$.
  \end{itemize}

  All functors preserve$\dots$
    \begin{itemize}
      \item isomorphisms. (\S3.21)
      \item sections (\S7.22)
      \item retractions (\S7.28)
    \end{itemize}

  All full, faithful functors reflect sections (\S7.23) and retractions
  (\S7.29).

  All representable functors preserve monos. (\S7.37.1)

  Faithful functors reflect monos (\S7.37.2) and epis (7.44).

\end{document}
