\documentclass[10pt,twocolumn,letterpaper]{article}
\DeclareSymbolFont{AMSb}{U}{msb}{m}{n}
\DeclareMathAlphabet{\mathbbm}{U}{bbm}{m}{n}
\title{Category Theory Cheat Sheet}
%\author{Nathaniel Wesley Filardo}

\usepackage{amsmath,amssymb,amsthm,latexsym}
\usepackage{fancyhdr}
\usepackage[tiny,center,compact,sc]{titlesec}
\usepackage[cm]{fullpage}
\usepackage{pstricks}
\usepackage{graphicx}
\usepackage{verbatim}
\usepackage{bm}
\usepackage{ifthen}
\usepackage{epsfig}
\usepackage[all]{xypic}
\usepackage{textcomp}
\usepackage{url}
\usepackage{multirow}
\usepackage{hyperref}
\usepackage{breakurl}

\renewcommand{\baselinestretch}{0.9}

%\newtheorem{thm}{Thm}[section]
%\newtheorem{dfn}{Def}[section]

\setlength{\parindent}{0pt}
\setlength{\parskip}{3pt}

%Scalable bracket-like
\newcommand{\paren}[1]{\left({#1}\right)}
\newcommand{\brak}[1]{\left[{#1}\right]}
\newcommand{\abs}[1]{\left\lvert{#1}\right\rvert}
\newcommand{\ang}[1]{\left\langle{#1}\right\rangle}
\newcommand{\set}[1]{\left\{{#1}\right\}}

%Mathematics
\newcommand{\condexp}[1]{\ifthenelse{\equal{#1}{false}}{}{^{#1}}}
\newcommand{\dd}[3][false]{\frac{d\condexp{#1}{#2}}{d{#3}\condexp{#1}}}
\newcommand{\pd}[3][false]{\frac{\partial\condexp{#1}{#2}}{\partial{#3}\condexp{#1}}}

\newcommand{\ifrac}[2]{{#1}/{#2}}

%Quantum Mechanics
\newcommand{\ket}[1]{\left\lvert{#1}\right\rangle}
\newcommand{\bra}[1]{\left\langle{#1}\right\rvert}
\newcommand{\braket}[2]{\left\langle{#1}\middle\vert{#2}\right\rangle}
\newcommand{\Braket}[3]{\left\langle{#1}\middle\vert{#2}\middle\vert{#3}\right\rangle}
\newcommand{\dyad}[2]{\left\lvert{#1}\middle\rangle\middle\langle{#2}\right\rvert}

\DeclareMathOperator{\mm}{\mid\mid}

\newcommand{\defn}[1]{{\bf #1}}

\begin{document}

Notation and references are to Ji\v{r}\'i Ad\'amek, Horst Herrlich, George
E. Strecker's {\em Abstract and Concrete Categories: The Joy of Cats}.

\section{Basics}

  A \defn{category} (\S3.1) is a quadruple
  $(\mathcal{O},\mbox{hom},id,\circ)$ with
  \begin{itemize}
    \item A collection of objects $\mathcal{O}$
  \item For each object $A,B$, a (disjoint) collection of arrows
    $\mbox{hom}(A,B)$ (from \defn{domain} to \defn{codomain}).
    \item An associative arrow composition operator $\circ$.
    \item Identity arrows ($id_A$) on each object $A$, unit of $\circ$
  \end{itemize}

  The \defn{dual} (\S3.5) category $\mathbf{A}^\text{op}$ which
    exchanges domains and codomains of arrows in $\mathbf{A}$.

\section{Special Relations on Categories}

\section{Special Kinds of Arrows}

  $f : A \to B$ is an \defn{isomorphism} if $\exists!_g . f \circ g = id_B
    ~\wedge~ g \circ f = id_A$. (\S3.8; ! in \S3.11)

  Let $G$ be a functor $\mathbf{A} \to \mathbf{B}$ and $B$ a
  $\mathbf{B}$-object. A \defn{$G$-structured arrow with domain $B$}
  is a pair $(f : B \to GA, A)$.  It is
  \begin{itemize}
    \item \defn{generating} if $\forall_{r,s : A \to A'} . Gr \circ f = Gs
      \circ f \implies r = s$
    \item \defn{extremally generating} if it is generating and $\forall_{m :
      A' \to A, m ~\text{mono}, (g,A')} . f = Gm \circ g \implies m ~\text{iso}$.
    \item \defn{$G$-universal for $B$} if $\forall_{(f', A')} .
    \exists!_{\check f} . f' = G{\check f} \circ f$.
  \end{itemize}

\section{Functors}

  A \defn{(covariant) functor} $F : \mathbf{A} \to \mathbf{B}$ (\S3.17)
  assigns to each $\mathbf{A}$-object a $\mathbf{B}$-object and to each
  $\mathbf{A}$-morphism a $\mathbf{B}$-morphism s.t. composition and
  identites are {\em preserved}.

  A \defn{contravariant functor} $F : \mathbf{A} \to \mathbf{B}$ (\S3.20.5)
  is a functor from $\mathbf{A}^\text{op} \to \mathbf{B}$.

  All functors preserve isomorphisms. (\S3.21)

  A functor is (\S3.27)
  \begin{itemize}
    \item an \defn{embedding} if it is injective on morphisms.
    \item \defn{faithful} if $\forall_{A,A'}$ the restriction $F\vert_{\mbox{hom}_A(A,A')}
      \subseteq \mbox{hom}(FA, FA')$ is injective.
    \item \defn{full} if said restrictions are surjective.
    \item \defn{amnestic} if $f$ is an identity iff $Ff$ is an identity.
  \end{itemize}

  A functor $F : \mathbf{A} \to \mathbf{B}$ is (\S3.33)
  \begin{itemize}  
    \item \defn{isomorphism-dense} if $\forall_B . \exists_A . F(A) \simeq B$.
    \item an \defn{equivalence} if it is full, faithful, and
      isomorphism-dense.
  \end{itemize}

\end{document}
