\documentclass[10pt,twocolumn,letterpaper]{amsart}
\DeclareSymbolFont{AMSb}{U}{msb}{m}{n}
\DeclareMathAlphabet{\mathbbm}{U}{bbm}{m}{n}
\title{Category Theory Cheat Sheet}
%\author{Nathaniel Wesley Filardo}

\usepackage{amsmath,amssymb,amsthm,latexsym}
\usepackage{fancyhdr}
\usepackage[cm]{fullpage}
\usepackage{pstricks}
\usepackage{graphicx}
\usepackage{verbatim}
\usepackage{bm}
\usepackage{ifthen}
\usepackage{epsfig}
\usepackage[all]{xypic}
\usepackage{textcomp}
\usepackage{url}
\usepackage{multirow}
\usepackage{hyperref}
\usepackage{breakurl}

\renewcommand{\baselinestretch}{0.9}

\newtheorem{thm}{Thm}[section]
\newtheorem{dfn}{Def}[section]

%Scalable bracket-like
\newcommand{\paren}[1]{\left({#1}\right)}
\newcommand{\brak}[1]{\left[{#1}\right]}
\newcommand{\abs}[1]{\left\lvert{#1}\right\rvert}
\newcommand{\ang}[1]{\left\langle{#1}\right\rangle}
\newcommand{\set}[1]{\left\{{#1}\right\}}

%Mathematics
\newcommand{\condexp}[1]{\ifthenelse{\equal{#1}{false}}{}{^{#1}}}
\newcommand{\dd}[3][false]{\frac{d\condexp{#1}{#2}}{d{#3}\condexp{#1}}}
\newcommand{\pd}[3][false]{\frac{\partial\condexp{#1}{#2}}{\partial{#3}\condexp{#1}}}

\newcommand{\ifrac}[2]{{#1}/{#2}}

%Quantum Mechanics
\newcommand{\ket}[1]{\left\lvert{#1}\right\rangle}
\newcommand{\bra}[1]{\left\langle{#1}\right\rvert}
\newcommand{\braket}[2]{\left\langle{#1}\middle\vert{#2}\right\rangle}
\newcommand{\Braket}[3]{\left\langle{#1}\middle\vert{#2}\middle\vert{#3}\right\rangle}
\newcommand{\dyad}[2]{\left\lvert{#1}\middle\rangle\middle\langle{#2}\right\rvert}

\DeclareMathOperator{\mm}{\mid\mid}

\newcommand{\defn}[1]{\label{dfn:#1}{\em #1}}

\begin{document}

%\maketitle
\bibliographystyle{plainurl}

\section{Basics}

  \begin{dfn}A \defn{category} (p4,\S1.3) is a structure with
  \begin{itemize}
    \item Objects \& arrows (from \defn{domain} to \defn{codomain}).
    \item An associative arrow composition operator $\circ$.
    \item Identity arrows ($1_A$) on each object $A$, unit of $\circ$
  \end{itemize}
  \end{dfn}

  \begin{dfn}A \defn{functor} (p8,D1.2) $F$ is a map between categories which
     sends $A \to B$ to $FA \to FB$, sends $1_A$ to $1_{FA}$, and honors composition.
  \end{dfn}

  \begin{dfn}$\mbox{Hom}_\mathbf{C}(A,B)$ denotes the class $(A \to B) \in \mathbf{C}$.\end{dfn}

  \begin{dfn}The \defn{dual} (p15,i2) category $\mathbf{C}^{op}$ which
    exchanges domains and codomains of arrows in $\mathbf{C}$.
  \end{dfn}

  \begin{thm}Any CT statement implies its dual (interchange dom/cod and reverse compositions).\end{thm}

  \subsection{Categories over $\mathbf{C}$'s objects}

  \begin{dfn}The \defn{arrow} (p16,i3) category $\mathbf{C}^\to$ has arrows
     for commutative squares in $\mathbf{C}$.  There are two functors
     \[\xymatrix{ \mathbf{C} & \ar[l]_{\mathbf{dom}} C^\to \ar[r]^{\mathbf{cod}} & \mathbf{C}}\]
  \end{dfn}

  \begin{dfn}The \defn{slice} (p16,i4) category $\mathbf{C}/C$ has objects
     of arrows in $\mathbf{C}$ with codomain $C$.  Arrows are tops of commutative triangles.
  \end{dfn}

  \subsection{Foundations}

  \begin{thm}Categories may be described (p21) as
     \[\xymatrix{ C_2 \ar[r]^\circ & C_1 \ar@<2ex>[r]_{cod} \ar@<-2ex>[r]_{dom} & C_0 \ar[l]^i }\]
  \end{thm}

  \begin{dfn}A category is (p24-25,D1.11-12)\dots
    \begin{itemize}
      \item \defn{small} if $C_0$ and $C_1$ is a set and \defn{large} otherwise.
      \item \defn{locally small} if $\forall_{X,Y \in C_0} \mbox{Hom}_C(X,Y) \subseteq C_1$ is a set.
    \end{itemize}
  \end{dfn}

\section{Special Kinds of Arrows}

  \begin{dfn}$m$ is \defn{monic} (p25,D2.1) if $mi = mj \Rightarrow i = j$ in
    \[\xymatrix{C \ar@<1ex>[r]^{i} \ar@<-1ex>[r]_j & A \ar@{>->}[r]^m & B} \]
  \end{dfn}

  \begin{dfn}A \defn{subobject} (p77,D5.1) of $X$ is mono with cod $X$.\end{dfn}

  \begin{dfn}$e$ is \defn{epic} (p25,D2.1) if it is monic in $\mathbf{C}^{op}$,
          {\it i.e.,} if $ie = je \Rightarrow i = j$ in
    \[\xymatrix{A \ar@{->>}[r]^e & B \ar@<1ex>[r]^{i} \ar@<-1ex>[r]_j & C} \]
  \end{dfn}

  \begin{thm}(p27,P2.6) Every iso is both monic and epic.\end{thm}

  \begin{dfn}A \defn{split mono} (\defn{epi}) has a left (right) inverse. (p28,D2.7)\end{dfn}

  \begin{dfn}Given $e : A \to B$ and $s : B \to A$ s.t. $es = 1_A$, $e$ is a \defn{retraction}
    of $s$ and $s$ is a \defn{section} (\defn{splitting}) of $e$. (p28,D2.7)\end{dfn}

  \begin{thm}Functors preserve split monos and epis.\end{thm}

   \begin{dfn}An \defn{point} (p32) of $C$ is any $c : 1 \to C$.\end{dfn}
                  
   \begin{thm}Arrows in $\mathbf{Sets}$, but not $\mathbf{Pos}$, are pointwise.\end{thm}

\section{Universal Constructions}

  \begin{thm}Objects defined by universal constructions are unique up to isomorphism.
  \end{thm}

  \begin{dfn}$0$ is \defn{initial} iff
    $\forall_C \exists! u : 0 \to C$.
  \end{dfn}

  \begin{dfn}$1$ is \defn{terminal} iff
    $\forall_C \exists! u : C \to 1 $.
  \end{dfn}

  \begin{dfn}$(A \times B,\pi_1,\pi_2)$ is a \defn{product} iff
    \[\xymatrix{
    {}\save[]+<-1cm,0cm>*\txt<8pc>{$\forall_{Z,z_1,z_2} \exists!_u$\\$u\pi_1 = z_1 ~\wedge~ u\pi_2 = z_2$}\restore
    &      & Z \ar[dl]_{z_1} \ar[dr]^{z_2} \ar[d]^u & \\
    &    A & \ar[l]_{\pi_1} A \times B \ar[r]^{\pi_2} & B \\
    }\]
  \end{dfn}

  \begin{dfn}$(E,e)$ is an \defn{equalizer} (p56,D3.13) of $f,g$ iff
     \[\forall_{Z,z . zf = zg} \exists!_u eu = z \quad
     \xymatrix{
     Z \ar@{..>}[r]^u \ar@/_1pc/[rr]^{z} & E \ar[r]^e & A \ar@<1ex>[r]^f \ar@<-1ex>[r]_g & B \\
     }\]
  \end{dfn}

  \begin{dfn}$(P,p_1,p_2)$ is a \defn{pullback} (p80,D5.4) of $f,g$ iff
     \[\xymatrix{
     {}\save[]+<-1cm,0cm>*\txt<8pc>{$\forall_{Z,z_1,z_2 . fz_1 = gz_2}\exists!_u$\\$z_1 = p_1u ~\wedge~ z_2 = p_2u$}\restore
     & Z \ar[dr]_{z_2} \ar@/^1pc/[rr]_{z_1} \ar[r]_u & P \ar[d]^{p_1} \ar[r]_{p_2} & B \ar[d]^g \\
     &                                              & A \ar[r]^f & C
     }\]
     $P$ may be denoted $A \times_C B$ when $f,g$ are clear.
  \end{dfn}

\section{Properties of UCs}

  \begin{thm}Equalizers are monic.\end{thm}

  \begin{thm}(p81,P5.5) $(Z,u)$ in a pullback is an equalizer of $fp_1$ and $gp_2$.
    If $(E,e)$ is an equalizer of same, then $E,p_1e,p_2e$ is a pullback of $f,g$.
  \end{thm}

  \begin{thm}(p84,L5.8) In the commuting diagram
    \[\xymatrix{ F \ar[r]_{f'} \ar[d]^{h''} & E \ar[r]_{g'} \ar[d]^{h'} & D \ar[d]^{h} \\
       A \ar[r]^f & B \ar[r]^g & C
    }\]
    \begin{enumerate}
      \item If $FEBA$ and $EDCB$ are pullbacks, so is $FDCA$.
      \item If $FDCA$ and $EDCB$ are pullbacks, so is $FEBA$.
    \end{enumerate}
  \end{thm}

  \begin{thm}Pullbacks preserve commutative triangles.\end{thm}

\section{Special Functors}

  \begin{dfn}The \defn{covariant representable functor} (p44) is
     \[\mbox{Hom}(A,\text{---}) : \mathbf{C} \to \mathbf{Sets}\]
  \end{dfn}

  \begin{thm}(p85,P5.10) Pullback defines a functor
    \[ h^* : (A \stackrel{\alpha}{\to} C) \in \mathbf{C}/C
       \mapsto (C' \times_C A \stackrel{\alpha'}{\to} C') \in \mathbf{C}/C' \]
    where $\alpha'$ is the pullback of $\alpha$ along $h$.
  \end{thm}

\section{Glossary}

  \begin{dfn}A category is \defn{finitely presented} (p75) if it is the
  free category over a finite graph quotiented by a finite set of equations.
  \end{dfn}

  \begin{dfn}
  A structure is \defn{free} over $S$ if its elements are ``generated''
  from $S$ and no ``nontrivial'' equations exist.
  \end{dfn}

  \begin{dfn}Subobject $m$'s \defn{local membership relation}:
          \[ \forall_{m : M \rightarrowtail X}
             \brak{ z \in_X M \Leftrightarrow \exists_{f:Z \to M} z = mf} \] 
  \end{dfn}

\end{document}

% vim:ts=2:expandtab
