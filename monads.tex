\documentclass[10pt,letterpaper]{article}
\DeclareSymbolFont{AMSb}{U}{msb}{m}{n}
\DeclareMathAlphabet{\mathbbm}{U}{bbm}{m}{n}

\usepackage{amsmath,amssymb,amsthm,latexsym}
\usepackage{fancyhdr}
\usepackage[tiny,center,compact,sc]{titlesec}
\usepackage[cm]{fullpage}
\usepackage{pstricks}
\usepackage{graphicx}
\usepackage{verbatim}
\usepackage{bm}
\usepackage{ifthen}
\usepackage{epsfig}
\usepackage[all]{xypic}
\usepackage{textcomp}
\usepackage{url}
\usepackage{multirow}
\usepackage{hyperref}
\usepackage{breakurl}
\usepackage{listings}

\renewcommand{\baselinestretch}{0.9}

%\newtheorem{thm}{Thm}[section]
%\newtheorem{dfn}{Def}[section]

\setlength{\parindent}{0pt}
\setlength{\parskip}{3pt}

%Scalable bracket-like
\newcommand{\paren}[1]{\left({#1}\right)}
\newcommand{\brak}[1]{\left[{#1}\right]}
\newcommand{\abs}[1]{\left\lvert{#1}\right\rvert}
\newcommand{\ang}[1]{\left\langle{#1}\right\rangle}
\newcommand{\set}[1]{\left\{#1\right\}}

%Mathematics
\newcommand{\condexp}[1]{\ifthenelse{\equal{#1}{false}}{}{^{#1}}}
\newcommand{\dd}[3][false]{\frac{d\condexp{#1}{#2}}{d{#3}\condexp{#1}}}
\newcommand{\pd}[3][false]{\frac{\partial\condexp{#1}{#2}}{\partial{#3}\condexp{#1}}}

\newcommand{\ifrac}[2]{{#1}/{#2}}

%Quantum Mechanics
\newcommand{\ket}[1]{\left\lvert{#1}\right\rangle}
\newcommand{\bra}[1]{\left\langle{#1}\right\rvert}
\newcommand{\braket}[2]{\left\langle{#1}\middle\vert{#2}\right\rangle}
\newcommand{\Braket}[3]{\left\langle{#1}\middle\vert{#2}\middle\vert{#3}\right\rangle}
\newcommand{\dyad}[2]{\left\lvert{#1}\middle\rangle\middle\langle{#2}\right\rvert}

\DeclareMathOperator{\mm}{\mid\mid}

\newcommand{\defn}[1]{{\bf #1}}
\newcommand{\natto}{\overset{\cdot}{\to}}

\lstset{mathescape=true}

\begin{document}

\section{Expanding The Diagrams}

The definition and diagrams given in ACC (\S20.1) are sort of terse, so I
have taken the liberty of applying the notation in (\S6.2) and (\S3.23,
footnote 15) and applying the functors to particular objects $X,Y \in
\mathbf{X}$:
\begin{itemize}
  \item The definition now reads as: A \defn{monad} on $\mathbf{X}$ is $(T : \mathbf{X} \to \mathbf{X},
  \eta : id_{\mathbf{X}} \natto T, \mu : T^2 \natto T)$ s.t.
  \[\forall_X \quad
  \xymatrix{
    T^3X \ar[r]^{T(\mu_X)} \ar[d]^{\mu_{TX}} & T^2X \ar[d]^{\mu_{X}} \\
    T^2X \ar[r]^{\mu_X}                 & TX
  } \quad \xymatrix{
    TX \ar[r]^{T(\eta_X)} \ar[dr]_{id_{TX}} & T^2X \ar[d]^{\mu_X} & TX \ar[l]_{\eta_{TX}} \ar[dl]^{id_{TX}} \\
        & TX & 
  }\]
  \item The naturality conditions unpack to be
  \[\forall_{X,Y,f} \quad \xymatrix{
    X \ar[r]^{\eta_X} \ar[d]^f & TX \ar[d]^{Tf} \\
    Y \ar[r]^{\eta_Y}          & TY
  }\quad\xymatrix{
    T^2X \ar[r]^{\mu_X} \ar[d]^{T^2f} & TX \ar[d]^{Tf} \\
    T^2Y \ar[r]^{\mu_Y}               & TY
  }\]
\end{itemize}

\section{Translation into Haskell}

\url{http://en.wikibooks.org/wiki/Haskell/Category_theory#The_monad_laws_and_their_importance}
may be of use, and I am going to try a brief, more equational, exposition
here (with many more parens than strictly necessary; deal with it).

$\eta$ pretty clearly corresponds to \texttt{return}, and $Tf$ is
\texttt{fmap f}.  The naturality condition on $\eta$ is clear:
\begin{lstlisting}[language=Haskell]
    fmap f . return  -- top right
=== return . f       -- bottom left
\end{lstlisting}
This is properly read as a constraint (part of the definition) of
\texttt{return} ($\eta$) in terms of \texttt{fmap} applied at the type
(constructor / functor) associated with our monad (i.e., the morphism part
of the functor).

Similarly, $\mu$ corresponds to \texttt{join}, whose Haskell definition
is
\begin{lstlisting}
join :: m (m a) -> m a
join mma = (mma >>= id)   -- or just "join = (>>= id)"
\end{lstlisting}
(Haskell, by convention, uses \texttt{m} for $T$; sorry for the confusion.)
Its naturality condition says that
\begin{lstlisting}
    (fmap f) . join         -- top right
=== join . (fmap (fmap f))  -- bottom left
\end{lstlisting}
This says, basically, that you can first run your inner monadic thingie and
then apply a ``lifted'' function to the result, or you can lift the function
twice, so that it applies inside your inner monadic thingie and {\em then}
run the inner thing.  Again, this should be taken as a constraint (part of
the definition) on join ($\mu$) in terms of \texttt{fmap}.

And now the other two laws, which are more interesting and can nicely be
executed in terms of Haskell's \verb|>>=|.  First, we have:
\[ \mu_X \circ T(\mu_X) = \mu_X \circ \mu_{TX} \]
or
\begin{lstlisting}
join . (fmap join) === join . join
\end{lstlisting}
Which is easy enough to see:
\begin{lstlisting}
  join (fmap join mmmx)
= (mmmx >>= return . join) >>= id                 -- defn  join, fmap
= (mmmx >>= (\mmx -> return (join mmx))) >>= id   -- syntax
= (mmmx >>= (\mmx -> return (mmx >>= id))) >>= id -- defn  join
= mmmx >>= (\mmx -> (return (mmx >>= id) >>= id)) -- assoc >>=
= mmmx >>= (\mmx -> id (mmx >>= id))              -- left-identity >>=
= mmmx >>= (\mmx -> mmx >>= id)                   -- apply
= mmmx >>= (\mmx -> mmx) >>= id                   -- assoc >>=
= (mmmx >>= id) >>= id                            -- assoc >>=
= join (join mmmx)                                -- defn  join, join
\end{lstlisting}
If we label our \text{mmmx} object as $\mathtt{m_1 m_2 m_3 x}$, this
says, in some pseudo-notation, that $\mathtt{join (m_1 (m_{23} x))}$
is the same as $\mathtt{join (m_{12} (m_3 x))}$.

And for the last, we have:
\[ id = \mu_X \circ T(\eta_X) = \mu_X \circ \eta_{TX} \]
or
\begin{lstlisting}
id === join . (fmap return) === \tx -> join . (return tx)
\end{lstlisting}
(note that we do not interpret $\eta_{TX}$ as \texttt{return . return}, but
as \texttt{return tx}!  There's no guarantee that a (generalized) element of
$TX$ is the result of $\eta_X$ -- consider, for example, the \texttt{Either
e} (i.e., $(e+)$) monads!)
which again admits executable rewriting (I have taken the liberty of
subscripting some functions, just to make the rewrites clearer.)
\begin{lstlisting}
  join (fmap return tx)
= join (tx >>= return$_1$ . return$_2$)                   -- defn  fmap
= (tx >>= return$_1$ . return$_2$) >>= id                 -- defn  join
= (tx >>= (\x -> return$_1$ (return$_2$ x))) >>= id       -- syntax
= tx >>= (\x -> ((return$_1$ (return$_2$ x)) >>= id)      -- assoc >>=
= tx >>= (\x -> id (return$_2$ x))                    -- left-identity >>=
= tx >>= (\x -> return$_2$ x)                         -- apply
= tx >>= return$_2$                                   -- syntax
= tx                                              -- right-identity >>=
\end{lstlisting}
and
\begin{lstlisting}
  \tx -> join . (return tx)
= \tx -> ((return tx) >>= id)                     -- defn  join
= \tx -> (id tx)                                  -- left-identity >>=
= \tx -> tx                                       -- apply
= id                                              -- defn
\end{lstlisting}

\section{Speaking of Haskell}

For the curious, \verb|>>=| can be implemented in terms of join (which makes
the above arguments circular (sorry!), but puts Haskell's typical treatment
of Monads on firmer ground):
\begin{lstlisting}
(>>=) :: m a -> (a -> m b) -> m b
ma >>= f  = join (fmap f ma)
--        = (fmap f ma) >>= id
--        = (ma >>= return . f) >>= id
--        = ma >>= (\a -> (return (f a) >>= id))
--        = ma >>= (\a -> id (f a))
--        = ma >>= \a -> f a
--        = ma >>= f
\end{lstlisting}

\end{document}
